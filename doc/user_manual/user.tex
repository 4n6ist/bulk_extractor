\subsection{lift}
LIFT is the Carnegie Mellon University Learning to Identify File
Types, a machine learning approach for identifying file fragment types
that uses support vector machines (SVMs) for classification and
character n-grams for the features that are classified.

The LIFT scanner supports subcommands that are passed to the scanner
through configuration settings. The following settings are used:

\begin{compactdesc}
\item[liftcmd] Specifies the subcommand to be executed.
\item[liftif] Specifies the LIFT input file.
\item[liftof] Specifies the OF file.
\end{compactdesc}

The LIFT scanner supports the following subcommands:

\begin{compactdesc}
\item[help] prints information about the subcommands.
\item[info] prints information about the currently loaded
  classification model.
\item[a2c] transforms an ASCII classification model on the liftif to
  be transformed into a compiled classification model on the liftof.
\end{compactdesc}

Typically subcommands are run with only the LIFT scanner, as specified
with the -E option. Thus run the info subcommand, one would specify:

\begin{Verbatim}
$ bulk_extractor -E lift -s liftcmd=info
\end{Verbatim}
